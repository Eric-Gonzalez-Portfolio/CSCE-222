\documentclass[12pt]{article}
\input xypic


\begin{document}
\begin{center}
\textbf{How to draw a derivation tree and a state table in LaTex}
\end{center}

\thispagestyle{empty}

\noindent
1. For example, with the grammar in Exercise 24, Section 13.1 (page 857), the derivation tree for  \textsl{abbccb} is given by
$$
\xymatrix{
&& S \ar[dll] \ar[dl] \ar[d] \\ 
a & b & S \ar[dll] \ar[dl] \ar[d] \\ 
b & c & S \ar[dl] \ar[d] \\ 
& c & b 
}
$$


\bigskip


\noindent
2. For example, the state table given in Exercise 1 a) in Section 13.2 (page 863) can be given by
$$
\begin{tabular}{c|cc|cc|}  % the table has five columns
& \multicolumn{2}{c|}{f} & \multicolumn{2}{c|}{g} \\  % f is over the second and third columns and g over the fourth and fifth columns
\cline{2-5}   % put a line over the columns from the second to the fifth
& \multicolumn{2}{c|}{Input} & \multicolumn{2}{c|}{Input} \\
State & 0 & 1 & 0 & 1 \\
\hline
$s_0$ & $s_1$ & $s_0$ & 0 & 1 \\
$s_1$ & $s_0$ & $s_2$ & 0 & 1 \\
$s_2$ & $s_1$ & $s_1$ & 0 & 0
\end{tabular}
$$

\end{document}
