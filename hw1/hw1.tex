\documentclass{article}
\usepackage{amsmath,amsthm,latexsym,paralist}

\theoremstyle{definition}
\newtheorem{problem}{Problem}
\newtheorem*{solution}{Solution}
\newtheorem*{resources}{Resources}

\newcommand{\name}[1]{\noindent\textbf{Name: Eric E. Gonzalez}}
\newcommand{\honor}{\noindent On my honor, as an Aggie, I have neither
  given nor received any unauthorized aid on any portion of the
  academic work included in this assignment. Furthermore, I have
  disclosed all resources (people, books, web sites, etc.) that have
  been used to prepare this homework. \\[2ex]
 \textbf{Signature:} \underline{\hspace*{7cm}} }

 
\newcommand{\checklist}{\noindent\textbf{Checklist:}
\begin{compactenum}
\item Did you add your name? 
\item Did you disclose all resources that you have used? \\
(This includes all people, books, websites, etc.\ that you have consulted)
\item Did you sign that you followed the Aggie honor code? 
\item Did you solve all problems? 
\item Did you submit (a) your latex source file and (b) the resulting pdf file
  of your homework on CSNet?
\item Did you submit (c) a signed hardcopy of the pdf file in class? 
\end{compactenum}
}

\newcommand{\problemset}[1]{\begin{center}\textbf{Problem Set #1}\end{center}}
\newcommand{\duedate}[2]{\begin{quote}\textbf{Due dates:} Electronic
    submission of hw1.tex and hw1.pdf files of this homework is due on
    \textbf{#1} on \texttt{https://csnet.cs.tamu.edu}.  A signed paper copy of the pdf file is due on
    \textbf{#2} at the beginning of class.\end{quote} }


\begin{document}
\begin{center}
{\large
CSCE 222-501 Discrete Structures for Computing\\[.5ex]
Fall 2014 -- Hyunyoung Lee\\}
\end{center}
\problemset{1}
\duedate{9/15/2014 (Monday) before 11:59 p.m.}{9/16/2014 (Tuesday)}
\name{ (type your name here) }
\begin{resources} (Discrete Mathematics and its Applications 7th Edition by Rosen)
\end{resources}
\honor

\bigskip

\begin{problem} ($2 \mbox{ pts} \times 5 = 10$ points)
Section 2.1, Exercise 8 b), c), d), e), and f), page 125
\end{problem}
\begin{solution} 
b) No c) Yes d) Yes e) Yes f) No
\end{solution}

\begin{problem} ($2 \mbox{ pts} \times 5 = 10$ points)
Section 2.1, Exercise 10 b), d), e), f), and g), page 125
\end{problem}
\begin{solution} 
b) True d) True e) True f) True g) False
\end{solution}

\begin{problem} (10 points)
Section 2.1, Exercise 26, page 126. Prove your answer. 
\end{problem}
\begin{solution} 
The Cartesian Product of sets A and B is A$\times$B = $\{$(a,b) $\mid$ a $\in$ A $\wedge$ b $\in$ B$\}$. That of C and D is C$\times$D = $\{$(c,d) $\mid$ c $\in$ C $\wedge$ d $\in$ D$\}$. If A$\subseteq$C and B$\subseteq$D, then the product of A and B must must be less than or equal to that of the larger sets C and D, and also include only elements that are already present in the two supersets. Therefore, we can conclude that A$\times$B $\subseteq$ C$\times$D.
\end{solution}

\begin{problem} ($2.5 \mbox{ pts} \times 4 = 10$ points)
Section 2.2, Exercise 2, page 136
\end{problem}
\begin{solution} 
a) $A \cap B$
\\b) $A - (A \cap B)$
\\c) $A \cup B$
\\d)$\overline{A} \cup \overline{B}$
\end{solution}

\begin{problem} ($3+3+4 = 10$ points)
Section 2.2, Exercise 16 a), b), and d), page 136
\end{problem}
\begin{solution} 
a) A$\subseteq$B = $\{$x$\mid$x $\in$ (A$\cap$B)$\}$ = $\{$x$\mid$x $\in$ A and B intersections$\}$ $\subseteq$ $\{$x$\mid$x $\in$ A$\}$
\\Therefore, (A$\cap$B)$\subseteq$A.
\\b) A$\cup$B forms a superset composed of all elements of A combined with all elements of B. As such, all of set A is included in A$\cup$B. Therefore, A $\subseteq$(A$\cup$B).
\\d) Because B - A results in a set containing all elements of B not present in A, there are no common elements left between A and (B - A) that intersect. Therefore, A $\cap$ (B - A) = $\emptyset$
\end{solution}

\begin{problem} ($5 \mbox{ pts} \times 2 = 10$ points)
Section 2.2, Exercise 50 b) and c), page 137
\end{problem}
\begin{solution} 
b) $\cup_{i=1}^\infty A_i = \{0,1\}\cup \{0,2\}\cup \{0,3\}\cup...\cup\{0,n\} = \{0,1,2,3,...\}=Z^+$
\\ $\cap_{i=1}^\infty A_i = \{0,1\}\cap \{0,2\}\cap \{0,3\}\cap... = \{0\}$
\\c)  $\cup_{i=1}^\infty A_i = (0,1)\cup (0,2)\cup (0,3)\cup...\cup(0,n) = (0,\infty)=Z^+$
\\$\cap_{i=1}^\infty A_i = (0,1)\cap (0,2)\cap (0,3)\cap (0,n)\cap... = (0,1)$
\end{solution}

\begin{problem} ($2.5 \mbox{ pts} \times 4 = 10$ points)
Section 2.3, Exercise 12, page 153
\end{problem}
\begin{solution} 
a) One-to-one
\\b) Not one-to-one
\\c) One-to-one
\\d) Not one-to-one
\end{solution}

\begin{problem} (10 points)
Section 2.3, Exercise 14, page 153
\end{problem}
\begin{solution} 
a) Onto
\\b) Not onto
\\c) Onto
\\d) Onto
\\e) Not onto
\end{solution}

\begin{problem} (10 points)
Section 2.3, Exercise 56, page 154
\end{problem}
\begin{solution} 
n = ($\lfloor b \rfloor$ - $\lceil a \rceil$) + 1
\end{solution}

\begin{problem} ($2.5 \mbox{ pts} \times 4 = 10$ points)
Section 2.3, Exercise 58, page 154
\end{problem}
\begin{solution} 
a) 4 bits will require one byte
\\b) 10 bits will require 2 bytes
\\c) 500 bits will require 63 bytes
\\d) 3000 bits will require 375 bytes
\end{solution}

\goodbreak
\checklist
\end{document}
