\documentclass{article}
\usepackage{amsmath,amssymb,amsthm,latexsym,paralist}

\theoremstyle{definition}
\newtheorem{problem}{Problem}
\newtheorem*{solution}{Solution}
\newtheorem*{resources}{Resources}

\newcommand{\name}[1]{\noindent\textbf{Name: #1}}
\newcommand{\honor}{\noindent On my honor, as an Aggie, I have neither
  given nor received any unauthorized aid on any portion of the
  academic work included in this assignment. Furthermore, I have
  disclosed all resources (people, books, web sites, etc.) that have
  been used to prepare this homework. \\[2ex]
 \textbf{Signature:} \underline{\hspace*{7cm}} }

 
\newcommand{\checklist}{\noindent\textbf{Checklist:}
\begin{compactitem}[$\Box$]
\item Did you add your name? 
\item Did you disclose all resources that you have used? \\
(This includes all people, books, websites, etc.\ that you have consulted)
\item Did you sign that you followed the Aggie honor code? 
\item Did you solve all problems? 
\item Did you submit (a) your latex source file and (b) the resulting pdf file
  of your homework on csnet?
\item Did you submit (c) a signed hardcopy of the pdf file in class? 
\end{compactitem}
}

\newcommand{\problemset}[1]{\begin{center}\textbf{Problem Set #1}\end{center}}
\newcommand{\duedate}[2]{\begin{quote}\textbf{Due dates:} Electronic 
    submission of hw2.tex and hw2.pdf files of this homework is due on \textbf{#1} on 
    csnet.cs.tamu.edu. A signed paper copy of the pdf file is due on \textbf{#2} 
    at the beginning of class. If you do not turn in the signed paper copy of the pdf
    file, your work will not be graded.\end{quote} }

\newcommand{\N}{\mathbf{N}}
\newcommand{\R}{\mathbf{R}}
\newcommand{\Z}{\mathbf{Z}}


\begin{document}
\vspace*{-15mm}
\begin{center}
{\large
CSCE 222-501 Discrete Structures for Computing\\[.5ex]
Fall 2014 -- Hyunyoung Lee\\}
\end{center}
\problemset{2}
\duedate{Wednesday 9/24/2014 before 11:59 p.m.}{Thursday 9/25/2014}
\name{ Eric E. Gonzalez }
\begin{resources} (Discrete Mathematics and its Applications 7th Edition by Rosen)
\end{resources}
\honor

\bigskip
\begin{problem} (10 points)
Nicely typeset the definitions of the Big Oh, Big Omega, Big Theta
asymptotic notations in \LaTeX, \textbf{as shown in class} (that is,
with the absolute values).  

{\small [Incidentally, I recommend that you keep a LaTeX file with
  all definitions and important theorems that we learn in this
  class. This will help you to memorize the definitions, and will
  allow you to quickly access this information when solving homework
  problems and studying for exams.] }
\end{problem}
\begin{solution}
$\break$
Big Oh: f $\in$ O(g) if and only if there exists some natural number $ n_0$ and a positive real constant U such that $|f(n)|$ $\le$ U$|g(n)|$ for all n where n $\ge$ $n_0$.
\\Big Omega: f(n) = $\Omega$(g(n)) if and only if there exists a positive constant L and a natural number $n_0$ such that L$|g(n)| \le |f(n)|$ holds for all n $\ge n_0$
\\Big Theta: f(n) = $\Theta$(g(n)) if and only if there exist positive real constants L and U and a natural number $n_0$ such that L$|g(n)| \le |f(n)| \le$ U$|g(n)|$
\end{solution}

\begin{problem}  (15 points)
Let $f_1, f_2, f_3, f_4$ be functions from the set $\N$ of natural numbers
to the set $\R$ of real numbers. Suppose that $f_1= O(f_2)$ and
$f_3=O(f_4)$. Use the definition of Big Oh \textit{given in class} to prove that 
$$f_1(n) + f_3(n) = O(\max(f_2(n),  f_4(n))).$$
\end{problem}
\begin{solution}
$\break$
$|f_1(n)|$ $\le$ $U_1|f_2(n)|$ holds for all $n \ge n_1$  and $|f_3(n)|$ $\le$ $U_2|f_4(n)|$ for all $n \ge n_2$.
\\$|f_1(n)| + |f_3(n)|$ $\le$  $U_1|f_2(n)|$ + $U_2|f_4(n)|$
\\$\le$ $U_1|f(n)|$ + $U_2|f(n)|$
\\= $(U_1 + U_2)|f(n)|$
\\= U$|f(n)|$
\\As such, the statement $f_1(n) + f_3(n) = O(\max(f_2(n),  f_4(n)))$ holds true according to Theorem 2 of Section 3.2.
\end{solution}

\begin{problem} (10 points)
Let $f_1, f_2, f_3$ be functions from the set $\N$ of natural numbers
to the set $\R$ of real numbers. Suppose that $f_1= \Omega(f_2)$ and
$f_2=\Omega(f_3)$. Is it possible that 
$$ f_1(n) < f_3(n)$$ 
holds for all natural numbers $n$? Give a proof or give an argument that this
is impossible. 
\end{problem}
\begin{solution}
$\break$
Let $f_1$ = 3n+1, $f_2$ = 4n+2, and $f_3$ = 5n + 4
\\$f_3$ - $f_1$ = 2n + 3
\\Therefore, $ f_1(n) < f_3(n)$ holds for all natural numbers of n.
\end{solution}

\begin{problem} (15 points) Determine which of the following
statements are correct. In each case, answer correct or incorrect, and justify your answer. 
\begin{compactenum}[(a)]
\item $ n^3 = O(n^2+n^3)$
\item $ n^3 = O(n^2+n\log n)$
\item $ n^3 = O(\frac{n^3}{2} + 100n^2)$
\end{compactenum}
\end{problem}
\begin{solution}
$\break$
(a) Correct. $|n^3|$ $\le$ U$|n^2 + n^3|$ satisfies all values n $\ge$ $n_0$ if U=1 and $n_0$=1.
\\(b) Incorrect. There are no values for U and $n_0$ that satisfy $|n^3|$ $\le$ U$|n^2 + n^3|$ for all values n $\ge$ $n_0$.
\\(c) Correct.  $|n^3|$ $\le$ U$|n^3/2 + 100n^2|$ satisfies all values n $\ge$ $n_0$ if U=2 and $n_0$=1.
\end{solution}


\begin{problem} (10 points)
Prove that 
\begin{compactenum}[(a)] 
\item $n^n = \Omega(2^n)$ holds, 
\item and that $2^n = \Omega(n^n)$ does not hold. 
\end{compactenum}
\end{problem}
\begin{solution}
$\break$
(a) L($2^n$) $\le$ $n^n$ when L = 1 and $n_0$ = 2. $n^n$ $\ge$ $2^n$ holds for all values n $\ge$ $n_0$. Therefore, $n^n = \Omega(2^n)$ holds.
\\(b) $2^n$ $\ge$ $n^n$ does not hold for all values n $\ge$ $n_0$. An example is that $2^3$ is not $\ge$ $3^3$. As such, no values for constant U or $n_0$ $>$ 2 satisfy the statement. Therefore, $2^n = \Omega(n^n)$ does not hold.
\end{solution}


\begin{problem}  (15 points) 
Does $\Theta(n^3+2n+1) = \Theta(n^3)$ hold?  Justify your answer.
\end{problem}
\begin{solution}
$\break$
$\Theta$ covers both upper and lower bounds of functions. The common upper bound value of $n^3$ between the two expressions 
links them such that $n^3$ = O($n^3 + 2n$ +1 ) and $n^3 + 2n$ +1 = O($n^3$). Therefore $\Theta(n^3+2n+1) = \Theta(n^3)$ holds.
\end{solution}

\begin{problem} (10 points) 
Let $k$ be a fixed positive integer. Show that 
$$ 1^k+2^k+\cdots + n^k  = O(n^{k+1}) $$
holds. 
\end{problem}
\begin{solution}
$\break$
No matter the value of k, the sum of the first n integers to the k-th power never surpasses $n^{k+1}$. Therefore, $|f(n)|$ $\le$ U$|g(n)|$ and the statement holds.
\end{solution}

\begin{problem}(15 points)
Suppose that you have two algorithms $A$ and $B$ that solve the same
problem. Algorithm $A$ has worst case running time $T_A(n) =
2n^2-2n+1$ and Algorithm $B$ has worst case running time $T_B(n) =
n^2+n-1$. 
\begin{compactenum}[(a)]
\item Show that both $T_A(n)$ and $T_B(n)$ are in $O(n^2)$ 
\item Show that $T_A(n) = 2n^2 + O(n)$ and $T_B(n) = n^2 +O(n)$. 
\item Explain which algorithm is preferable. 
\end{compactenum}
\end{problem}
\begin{solution}
$\break$
(a) $|2n^2-2n+1|$ $\le$ U$|n^2|$ where U=2 and $n_0$=1 for all n $\ge n_0$
\\$|n^2+n-1|$ $\le$ U$|n^2|$ where U=2 and  $n_0$=1 for all n $\ge n_0$
\\Therefore, $T_A(n)$ $\in$ $O(n^2)$ and $T_B(n)\in O(n^2)$.
\\
\\(b) 2$n^2$-2n+1 = 2$n^2$ + O(n) AND $n^2$ + n - 1 = $n^2$ + O(n)
\\-2n + 1 = O(n) AND n - 1 = O(n)
\\$|-2n+1|$ $\le$ U$|n|$ where U=1 and $n_0$=1
\\$|n-1|$ $\le$ U$|n|$ where U=1 and $n_0$=1
\\Therefore, $T_A(n)$ = 2$n^2$ + $O(n)$ and $T_B(n)$ = $n^2 + O(n)$.
\\
\\(c) $T_B(n)$ is preferable as it has the lesser of the two worst-case running times.
\end{solution}


\bigskip

\goodbreak
\checklist
\end{document}
