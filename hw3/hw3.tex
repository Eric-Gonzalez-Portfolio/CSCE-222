\documentclass{article}
\usepackage{amsmath,amssymb,amsthm,latexsym,paralist}

\theoremstyle{definition}
\newtheorem{problem}{Problem}
\newtheorem*{solution}{Solution}
\newtheorem*{resources}{Resources}

\newcommand{\name}[1]{\noindent\textbf{Name: #1}}
\newcommand{\honor}{\noindent On my honor, as an Aggie, I have neither
  given nor received any unauthorized aid on any portion of the
  academic work included in this assignment. Furthermore, I have
  disclosed all resources (people, books, web sites, etc.) that have
  been used to prepare this homework. \\[2ex]
 \textbf{Signature:} \underline{\hspace*{5cm}} }

 
\newcommand{\checklist}{\noindent\textbf{Checklist:}
\begin{compactitem}[$\Box$] 
\item Did you add your name? 
\item Did you disclose all resources that you have used? \\
(This includes all people, books, websites, etc. that you have consulted)
\item Did you sign that you followed the Aggie honor code? 
\item Did you solve all problems? 
\item Did you submit (a) your latex source file and (b) the resulting pdf file
  of your homework on csnet?
\item Did you submit (c) a signed hardcopy of the pdf file in class? 
\end{compactitem}
}

\newcommand{\problemset}[1]{\begin{center}\textbf{Problem Set #1}\end{center}}
\newcommand{\duedate}[2]{\begin{quote}\textbf{Due dates:} Electronic
    submission of hw3.tex and hw3.pdf files of this homework is due on
    \textbf{#1} on csnet.cs.tamu.edu.  Please do not archive or
    compress the files.  A signed paper copy of the pdf file is due on
    \textbf{#2} at the beginning of class.
    \textbf{If you do not turn in a signed hardcopy, your work will not be graded.}\end{quote} }

\newcommand{\N}{\mathbf{N}}
\newcommand{\R}{\mathbf{R}}
\newcommand{\Z}{\mathbf{Z}}


\begin{document}
\vspace*{-15mm}
\begin{center}
{\large
CSCE 222-501 Discrete Structures for Computing\\[.5ex]
Fall 2014 -- Hyunyoung Lee\\}
\end{center}
\problemset{3}
\duedate{9/29/2014 before 23:59}{9/30/2014}
\name{Eric E. Gonzalez}
\begin{resources} (Discrete Mathematics and its Applications 7th Edition by Rosen)
\end{resources}
\honor

\bigskip

\begin{problem} (10 points) Section 3.3, Exercise 4 on page 229
\end{problem}
\begin{solution}
The loop iterates until i = $2^k$ and i $\ge$ n. As such, k = $log_2$n and the number of operations is $O(log_2 n)$
\end{solution}

\begin{problem} (15 points) Section 3.3, Exercise 14 on pages 230
\end{problem}
\begin{solution}
$\break$
(a) y = $a_2$ = 3
\\
\\i=1
\\y = 3*2+1=7
\\
\\i=2
\\y = 7*2+1=15
\\
\\(b) The while loop contains one addition and one multiplication looping 1 to n times. Therefore, there are n additions and n multiplications.
\end{solution}

\begin{problem} (15 points) Section 3.3, Exercise 16 on page 230
\end{problem}
\begin{solution}
$\break$
(a) $2^{(8.64*10^{15})}$
\\(b) 8.64*$10^{12}$ OR 8640000000000
\\(c) 92951600
\\(d) 2939387
\\(e) 205197
\\(f) $log_2$(8.64*$10^{15}$)
\\(g) $\frac{1}{2}$ * $log_2$(8.64*$10^{15}$)
\\(h)$log_2$($log_2$(8.64*$10^{15}$))
\end{solution}

\begin{problem} (10 points) Section 1.1, Exercise 6 on page 13
\end{problem}
\begin{solution}
$\break$
(a) True
\\(b) True
\\(c) False
\\(d) False
\\(e) False
\end{solution}

\begin{problem} (10 points) Section 1.1, Exercise 10 on page 13
\end{problem}
\begin{solution}
$\break$
(a) The election is not decided.
\\(b) The election is decided or the votes have been counted.
\\(c) The election is not decided and the votes have been counted.
\\(d) If the votes have been counted, then the election is decided.
\\(e) If the votes have not been counted, then the election is not decided.
\\(f) If the election is not decided, then the votes have not been counted.
\\(g) The election is decided if and only if the votes have been counted.
\\(h) The votes have not been counted, or the election is not decided and the votes have been counted.
\end{solution}

\begin{problem} (10 points) Section 1.1, Exercise 14 on pages 13--14
\end{problem}
\begin{solution}
$\break$
(a) r$\land$$\lnot$q
\\(b) p$\land$q$\land$r
\\(c) r$\rightarrow$p
\\(d)(p$\land$$\lnot$q)$\land$r
\\(e)(p$\land$q)$\rightarrow$r
\\(f) r$\leftrightarrow$(q$\lor$p)
\end{solution}

\begin{problem} (10 points) Section 1.1, Exercise 32 e) and f) on page 15
\end{problem}
\begin{solution}
\end{solution}
$\break$
\begin{displaymath}
\begin{array}{|c c|c|c|c|c|}
p & q & \lnot p & q \rightarrow \lnot p & p \leftrightarrow q & (q \rightarrow \lnot p) \leftrightarrow (p \leftrightarrow q)\\
\hline
T & T & F & F & T & F\\
T & F & F & T & F & F\\
F & T & T & T & F & F\\
F & F & T & T & T & T\\
\end{array}
\end{displaymath}
\\
\begin{displaymath}
\begin{array}{|c c|c|c|c|c|}
p & q & \lnot q & p \leftrightarrow q & p \leftrightarrow \lnot q & (p \leftrightarrow q) \oplus (p \leftrightarrow \lnot q)\\
\hline
T & T & F & T & F & T\\
T & F & T & F & T & T\\
F & T & F & F & T & T\\
F & F & T & T & F & T\\
\end{array}
\end{displaymath}
\begin{problem} (10 points) Section 1.2, Exercise 20 on page 23
\end{problem}
\begin{solution}
$\break$
A cannot be telling the truth, because then B's statement that he is a knave is made logically impossible by means of paradox. Therefore, A is the knave and B is the knight.
\end{solution}

\begin{problem} (10 points) Section 1.2, Exercise 28 on page 23
\end{problem}
\begin{solution}
$\break$
Each person's status as a knight, knave, or spy is dependent upon that of the person before them. As such, a chain occurs in which A must be lying in order for everyone to be a knight, knave and spy. So: A (the knave) is lying; B (the spy) is lying about A's status; and C (the knight) is telling the truth about B's status.
\end{solution}


\goodbreak
\checklist
\end{document}
