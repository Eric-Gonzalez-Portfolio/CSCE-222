\documentclass{article}
\usepackage{amsmath,amsthm,latexsym,paralist}

\theoremstyle{definition}
\newtheorem{problem}{Problem}
\newtheorem*{solution}{Solution}
\newtheorem*{resources}{Resources}

\newcommand{\name}[1]{\noindent\textbf{Name: #1}}
\newcommand{\honor}{\noindent On my honor, as an Aggie, I have neither
  given nor received any unauthorized aid on any portion of the
  academic work included in this assignment. Furthermore, I have
  disclosed all resources (people, books, web sites, etc.) that have
  been used to prepare this homework. \\[1.5ex]
 \textbf{Signature:} \underline{\hspace*{5cm}} }

 
\newcommand{\checklist}{\noindent\textbf{Checklist:}
\begin{compactitem}[$\Box$] 
\item Did you add your name? 
\item Did you disclose all resources that you have used? \\
(This includes all people, books, websites, etc.\ that you have consulted)
\item Did you sign that you followed the Aggie honor code? 
\item Did you solve all problems? 
\item Did you submit (a) your latex source file and (b) the resulting pdf file
  of your homework on csnet?
\item Did you submit (c) a signed hardcopy of the pdf file in class? 
\end{compactitem}
}

\newcommand{\problemset}[1]{\begin{center}\textbf{Problem Set #1}\end{center}}
\newcommand{\duedate}[2]{\begin{quote}\textbf{Due dates:} Electronic
    submission of hw4.tex and hw4.pdf files of this homework is due on
    \textbf{#1} on csnet.cs.tamu.edu.  Please do not archive or
    compress the files.  A signed paper copy of the pdf file is due on
    \textbf{#2} at the beginning of class.
    \textbf{If you do not turn in a signed hardcopy, your work will not be graded.}\end{quote} }


\begin{document}
\vspace*{-15mm}
\begin{center}
{\large
CSCE 222-501 Discrete Structures for Computing\\[.5ex]
Fall 2014 -- Hyunyoung Lee\\}
\end{center}
\problemset{4}
\duedate{10/13/2014 before 23:59}{10/14/2014}
\name{ Eric E. Gonzalez}
\begin{resources} (Discrete Mathematics and its Applications 7th Edition by Rosen)
\end{resources}
\honor

\bigskip
\bigskip

\begin{problem} (5 points)
Section 1.3, Exercise 10 (d), page 35
\end{problem}
\begin{solution}
$\break$
(D)\\
P = [(p $\lor$ q) $\land$ (p $\rightarrow$ r) $\land$ (q $\rightarrow$ r)] $\rightarrow$ r
\\
\begin{displaymath}
\begin{array}{|c|c|c|c|c|c|c|c|c|}
p &  q & r & p \lor q & p \rightarrow r & q \rightarrow r & (p \lor q) \land (p \rightarrow r) & (p \lor q) \land (p \rightarrow r) \land (q \rightarrow r) & P\\
\hline
T & T & T & T & T & T & T & T & T\\
T & T & F & T & F & F & F & F & T\\
T & F & T & T & T & T & T & T & T\\
T & F & F & T & F & T & F & F & T\\
F & T & T & T & T & T & T & T & T\\
F & T & F & T & T & F & T & F & T\\
F & F & T & F & T & T & F & F & T\\
F & F & F & F & T & T & F & F & T\\
\end{array}
\end{displaymath}
\end{solution}

\begin{problem} (15 points)
Section 1.3, Exercise 50, page 36 (you find the definition of functionally
complete on page 35). 
\end{problem}
\begin{solution}
$\break$
(A)\\
\begin{displaymath}
\begin{array}{|c|c|c|c|}
p &  p &\downarrow p & p \neg q \\
\hline
T & T & F & F\\
T & F & F & F\\
F & T & F & F\\
F & F & T & T\\
\end{array}
\end{displaymath}
\\ (B)
\begin{displaymath}
\begin{array}{|c|c|c|c|c|}
p &  q & p \downarrow q & (p \downarrow q) \downarrow (p \downarrow q) & (p \lor q) \\
\hline
T & T & F & T & T \\
T & F & F & T & T \\
F & T & F & T & T \\
F & F & T & F & F \\
\end{array}
\end{displaymath}
\\ (C) Therefore, $\{ \downarrow \}$ is a complete list of logical operators.
\end{solution}

\begin{problem} (10 points)
Section 1.4, Exercise 32, page 55
\end{problem}
\begin{solution}
$\break$
(a) All dogs have fleas. 
\\$\forall$x(dog(x) $\to$ fleas(x))
\\
\\Negation: $\forall$x(dog(x) $\to$ fleas(x)) $\equiv$ $\exists$x(dog(x) $\land$ fleas(x))
\\There is a dog that does not have fleas.
\\
\\(b) There is a horse that can add.
\\$\exists$x(horse(x) $\land$ add(x))
\\
\\Negation: $\neg\exists$x(horse(x) $\land$ add(x)) $\equiv$ $\forall$x(horse(x) $\to$ $\neg$add(x))
\\ No horse can add.
\\
\\(c) Every koala can climb.
\\$\exists$x(koala(x) $\to$ climb(x))
\\
\\Negation: $\forall$x(koala(x) $\to$ climb(x)) $\equiv$ $\exists$x (koala(x) $\land$ $\neg$climb(x))
\\There is a koala that cannot climb.
\\
\\d) No monkey can speak French. 
\\$\forall$x(monkey(x) $\to$ $\neg$speakFrench(x))
\\
\\Negation: $\neg \forall$x $\neg$(monkey(x) $\to$ $\neg$speakFrench(x)) $\equiv$ $\exists$x (monkey(x) $\land$
speakFrench(x))
\\There is a monkey that can speak French.
\\
\\e) There exists a pig that can swim and catch fish.
\\$\exists$x (pig(x) $\land$ swim(x) $\land$ catchFish(x))
\\
\\Negation: $\neg\exists$x(pig(x) $\land$ swim(x) $\land$ catchFish(x)) $\equiv$ 
\\ $\forall$x$\neg$(pig(x) $\land$ swim(x) $\land$ catchFish(x)) $\equiv$ 
\\ $\forall$x(pig(x) $\to$ $\neg$(swim(x) $\land$ catchFish(x)))
\\There is no pig that can swim and catch fish.
\\
\end{solution}

\begin{problem} (10 points)
Section 1.5, Exercise 6, pages 64--65
\end{problem}
\begin{solution}
$\break$
(A) Randy Goldberg is enrolled in CS 252.
\\(B) A student is enrolled in Math 695.
\\(C) Carol Sitea is enrolled in a class.
\\(D) A student is enrolled in Math 222 and CS 252.
\\(E) If a student is enrolled in a certain class, then another student is enrolled in that same class.
\\(F) If and only if a student is enrolled in a certain class, then another student is enrolled in that same class.
\end{solution}

\begin{problem} (10 points)
Section 1.6, Exercise 4, page 78
\end{problem}
\begin{solution}
$\break$
(A) Simplification
\\(B) Disjunctive syllogism
\\(C) Modus ponens
\\(D) Addition
\\(E) Hypothetical syllogism
\end{solution}

\begin{problem} (10 points)
Section 1.6, Exercise 8, page 78
\end{problem}
\begin{solution}
$\break$
Universal instantiation and Modus tollens.
\end{solution}

\begin{problem} (10 points)
Section 1.6, Exercise 14 (c) and (d), page 79
\end{problem}
\begin{solution}
$\break$
(C) Step 1: Universal instantiation
\\Step 2: Modus ponens
\\Step 3: Universal instantiation
\\Step 4: Modus ponens
\\
\\(D) Step 1: Existential instantiation
\\Step 2: Simplification
\\Step 3: Universal instantiation
\\Step 4: Modus ponens
\\Step 5: Simplification
\\Step 6: Conjunction
\\Step 7: Existential generalization
\end{solution}

\begin{problem} (10 points)
Section 1.7, Exercise 18, page 91
\end{problem}
\begin{solution}
$\break$
(A) Assume that n is odd. $3n + 2$ = $3(2k+1)$ + 2. 
\\= $6k + 3 + 2$
\\=$6k + 5$
\\=$6(k+4) + 1$
\\=2(3k+2) + 1
\\Which is odd.
\\Therefore, if $n$ is an integer, and $3n + 2$ is even, then $n$ is also even.
\\(B) Seeking a proof by contradiction, assume that $3n + 2$ is even and $n$ is odd. Since $n$ is odd, $3n + 2$ is also odd since the product of two odd numbers is odd and remains odd when added by two. Therefore, the assumption is false. As such, if $n$ is an integer, and $3n + 2$ is even, then $n$ is also even.
\end{solution}

\begin{problem} (10 points)
Let $n>1$ be an integer. \textsl{Prove by contradiction} that if $n$ is a perfect square, then
$n+3$ cannot be a perfect square. 
\end{problem}
\begin{solution}
$\break$
If n is a perfect square, then there must exist some integer $k$ such that $n=k^2$. \\
Seeking a contradiction, we assume that if n is a perfect square, n + 3 is also a perfect square.
\\n = $k^2$
\\n + 3 = $(k+1)^2$ = $k^2$ + 2k + 1
\\2k + 1 = 3
\\2k = 2
\\k = 1
\\n = $k^2$ = $(1)^2$ = 1
\\Since n = 1, n + 3 is not a perfect square because n $>$ 1 is needed for a perfect square. Statement holds by contradiction.
\end{solution}

\begin{problem} (10 points)
Prove by induction that
$$\sum_{i=0}^n 3^i = \frac{3^{n+1}-1}{2}$$
holds for every non-negative integer $n$.
\end{problem}
\begin{solution}
$\break$
Let $n$ = $k+1$
\\Induction Basis: We assume that 
$$\sum_{i=0}^{k+1} 3^0 = \frac{3^{0+1}-1}{2}$$
simplifies to $1=1$
\\
\\
Induction Step: Assume that 
$$\sum_{i=0}^{k+1} 3^i = \frac{3^{(k+1)+1}-1}{2}$$ which simplifies to: $$ \frac{3^{k+2}-1}{2} = \frac{3^{k+2}-1}{2}$$
As such, 1 = 1.
\\The statement holds for all non-negative integers $n$.
\end{solution}


\goodbreak
\checklist
\end{document}
