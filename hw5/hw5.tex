\documentclass{article}
\usepackage{amsmath,amsthm,latexsym,paralist}

\theoremstyle{definition}
\newtheorem{problem}{Problem}
\newtheorem*{solution}{Solution}
\newtheorem*{resources}{Resources}

\newcommand{\name}[1]{\noindent\textbf{Name: #1}}
\newcommand{\honor}{\noindent On my honor, as an Aggie, I have neither
  given nor received any unauthorized aid on any portion of the
  academic work included in this assignment. Furthermore, I have
  disclosed all resources (people, books, web sites, etc.) that have
  been used to prepare this homework. \\[1.5ex]
 \textbf{Signature:} \underline{\hspace*{7cm}} }

 
\newcommand{\checklist}{\noindent\textbf{Checklist:}
\begin{compactitem}[$\Box$] 
\item Did you add your name? 
\item Did you disclose all resources that you have used? \\
(This includes all people, books, websites, etc.\ that you have consulted)
\item Did you sign that you followed the Aggie honor code? 
\item Did you solve all problems? 
\item Did you submit (a) your latex source file and (b) the resulting pdf file
  of your homework on csnet?
\item Did you submit (c) a signed hardcopy of the pdf file in class? 
\end{compactitem}
}

\newcommand{\problemset}[1]{\begin{center}\textbf{Problem Set #1}\end{center}}
\newcommand{\duedate}[2]{\begin{quote}\textbf{Due dates:} Electronic
    submission of hw5.tex and hw5.pdf files of this homework is due on
    \textbf{#1} on csnet.cs.tamu.edu.  Please do not archive or
    compress the files.  A signed paper copy of the pdf file is due on
    \textbf{#2} at the beginning of class.\end{quote}}
    


\begin{document}
\vspace*{-15mm}
\begin{center}
{\large
CSCE 222-501 Discrete Structures for Computing\\[.5ex]
Fall 2014 -- Hyunyoung Lee\\}
\end{center}
\problemset{5}
\duedate{10/20/2014 before 23:59}{10/21/2014} 
\name{Eric E. Gonzalez}
\begin{resources} (Discrete Mathematics and its Applications 7th Edition by Rosen)
\end{resources}
\honor

\bigskip

\begin{problem}
Section 2.4, Exercise 6 (b), (c), (d), (g) and (h), pages 167--168
\end{problem}
\begin{solution} 
$\break$
(b) 1, 3, 6, 10, 15, 21, 28, 36, 45, 55 
\\(c) 1, 5, 19, 65, 211, 665, 2059, 6305, 19171, 58025 
\\(d) 1, 1, 1, 2, 2, 2, 2, 2, 3, 3 
\\(g) 1, 2, 2, 4, 8, 11, 33, 37, 148, 153
\\(h) 1, 2, 2, 2, 2, 3, 3, 3, 3, 3

\end{solution}

\begin{problem}
Section 2.4, Exercise 16 (c), (d), (e), (f) and (g), page 168
\end{problem}
\begin{solution} 
$\break$
(c) $a_n$ = $4 - \frac{n(n+1)}{2}$
\\(d) $a_n$ = $-2^{n+2}$ + 3
\\(e) $a_n$ = $2(n+1)!$
\\(f) $a_n$ = $3*2^{n}n!$
\\(g) $a_n$ = $\frac{2n-1+(-1)^{n-1}}{4} + 7(-1)^n$
\end{solution}

\begin{problem} 
Section 5.1, Exercise 6, page 329 (use mathematical induction)
\end{problem}
\begin{solution} 
$\break$
Basis: P(1) = $(1 + 1)! - 1 = 2 - 1 = 1$, which makes P(1) true.
\\
\\Inductive Step:
\\P(k) = (k + 1)! - 1
\\Assume P(k) is true.
\\P(k+1) = $(k + 1)! - 1 + (k+1)(k+1)!$
\\$= (k + 1)! - 1 + (k + 1)(k + 1)!$ 
\\$= (k + 2)(k + 1)! - 1 $
\\$= (k + 2)! - 1$ 
\\$= [(k + 1) + 1]! - 1$
\\Hence, true by mathematical induction.
\end{solution}

\begin{problem} 
Section 5.1, Exercise 8, page 329 (use mathematical induction)
\end{problem}
\begin{solution} 
$\break$
Basis: P(0) = $2(-7)^0=2$ \\ $(1-(-7)^1)/4=8/4=2$, which makes P(0) true
\\
\\Inductive Step:
\\Assume P(k) is true
\\P(k+1) = $\frac{(1 - (-7)^{k+1})}{4}$
\\
\\ $\frac{(1 - (-7)^{k+1})}{4}$ + $2(-7)^{k+1}$ = $\frac{(1 - (-7)^{k+2})}{4}$
\\$1 - (-7)^{k+1}$ + $8(-7)^{k+1}$ = $1 - (-7)^{k+2}$
\\$(-7)^{k+1} - 8(-7)^{k+1} = (-7)^{k+2}$         multiplied both sides by -1
\\$\frac{(-7)^{k+1} - 8(-7)^{k+1} = (-7)^{k+1}(-7)}{(- 7)^{k+1}}$
\\$1 - 8$ = $-7$
\\Hence, true by induction.
\end{solution}

\begin{problem} 
Section 5.1, Exercise 10, page 330 (use mathematical induction)
\end{problem}
\begin{solution} 
$\break$
(a) $\frac{1}{1*2}$ = $\frac{1}{1} - \frac{1}{2}$ = $\frac{2}{2} - \frac{1}{2}$ = $\frac{1}{2}$
\\$\frac{1}{1*2} + \frac{1}{2*3} +...+ \frac{1}{n*(n+1)}$ = $(\frac{1}{1} - \frac{1}{2}) + (\frac{1}{2} - \frac{1}{3})+...+(\frac{1}{n} - \frac{1}{n+1})$
\\=$\frac{1}{1} - \frac{1}{n+1}$
\\=$\frac{n+1-1}{n+1}$
\\=$\frac{n}{n+1}$
\\
\\(b) Basis: P(1) = $\frac{1}{1*2}$ = $\frac{1}{2}$
\\$\frac{1}{1+1} = \frac{1}{2}$ which makes P(1) true
\\
\\Inductive Step:
\\P(k) = $\frac{1}{1*2} + \frac{1}{2*3} +...+ \frac{1}{k*(k+1)} = \frac{k}{(k+1)}$
\\Assume P(k) is true.
\\P(k+1) = $(\frac{1}{1*2} + \frac{1}{2*3} +...+ \frac{1}{k*(k+1)}) + \frac{1}{(k+1)(k+2)}$
\\= $\frac{k}{(k+1)} + \frac{1}{(k+1)(k+2)}$
\\= $\frac{1}{(k+1)}[k + \frac{1}{(k+2)}]$
\\= $\frac{1}{(k+1)}[\frac{k^2 + 2k +1}{(k+2)}]$
\\= $\frac{1}{(k+1)}*\frac{(k+1)^2}{(k+2)}$
\\= $\frac{k+1}{k+2}$
\\= $\frac{k+1}{(k+1)+1}$
\\Hence, true by induction.
\end{solution}

\begin{problem} 
Section 5.1, Exercise 14, page 330 (use mathematical induction)
\end{problem}
\begin{solution} 
$\break$
Basis: P(1) = $1*2^1 = (1-1)*2^{1+1} + 2 = 2$
\\
\\Inductive Step:
\\P(k) = $(k-1)*2^{k+1} + 2$
\\P(k+1) = $(k-1)*2^{k+1} + 2 + (k+1)*2^{k+1}$
\\= $2^{k+1}[(k-1)+(k+1)]+2$
\\= $2^{k+1}[2k] + 2$
\\= $2^{k+2}*k + 2$
\\= $k*2^{k+2} + 2$
\\Hence, true by induction.
\end{solution}

\begin{problem} 
Section 5.1, Exercise 24, page 330 (use mathematical induction)
\end{problem}
\begin{solution} 
$\break$
Basis: P(1): $\frac{2 - 1}{2} \ge \frac{1}{2}$ which holds
\\
\\Inductive Step:
\\P(k): $\frac{1}{2k} \le \frac{1*3*5*...*2k-1}{2*4*6*...*2k}$
\\P(k+1): $\frac{1}{2(k+1)} \le \frac{1}{2k}*\frac{2(k+1)-1}{2(k+1)}$
\\ $\frac{1}{2k+2} \le \frac{2k+1}{2k(2k+2)}$
\\
\\ $\frac{1}{2(k+1)} = \frac{1}{2k} * \frac{2k}{(2k+2)}$
\\ $\le [\frac{1*3*5*...*2k-1}{2*4*6*...*2k}] * \frac{2k}{2k+2}$ ... (by assumption) 
\\ $\le [\frac{1*3*5*...*2k-1}{2*4*6*...*2k}] * \frac{2k+1}{2k+2}$
\\ $\le [\frac{1*3*5*...*(2k-1)*(2k+1)}{2*4*6*...*2k*(2k+2)}]$ 
\\ $\le [\frac{1*3*5*...*(2(k+1)-1)}{2*4*6*...*2(k+1)}]$
\\Hence, true by induction.
\end{solution}

\begin{problem} 
Section 5.1, Exercise 32, page 330 (use mathematical induction)
\end{problem}
\begin{solution} 
$\break$
Basis: P(1) = $1^3 + 2*1 = 3$ which holds true, since it is divisible by 3
\\
\\Inductive Step: 
\\P(k) =$ k^3 + 2k$
\\P(k+1) = $(k+1)^3 + 2(k+1)$
\\= $k^3 + 1^3 + 3k^2 + 3k + 2k + 2$
\\= $k^3 + 2k + 3k^2 + 3k + 3$
\\= $(k^3 + 2k) + 3(k^2 + k + 1)$ which is divisible by 3
\\Hence, true by induction.
\end{solution}

\begin{problem} 
Section 5.2, Exercise 12, page 342
\end{problem}
\begin{solution} 
$\break$
Basis: $1=2^0$ represents numbers in binary form.\\
By the inductive hypothesis, assume that every positive integer up to k can be represented as a sum of powers of 2.  If k + 1 is odd, then k is even. If k + 1 is even, then $(k + 1)/2$ is a positive integer, and can be written as a sum of distinct powers of two. After increasing every exponent by 1, it doubles the value and provides the sum for k+1.
\end{solution}

\begin{problem} 
Section 5.2, Exercise 30, page 344
\end{problem}
\begin{solution} 
$\break$
The flaw is that the inductive step requires that both $a^k = 1$
\\and $a^{k-1}$ = 1.  The base case has to be sufficient
\\to support the inductive hypothesis in the first step. Here it isn’t, 
\\as there are multiple values for which the statement is not supported. The statement breaks at k=1.
\end{solution}

\noindent
Before attempting the problems from Chapter 5, make sure that you have carefully
read Chapter 5.

\bigskip

\goodbreak
\checklist
\end{document}
