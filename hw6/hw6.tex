\documentclass{article}
\usepackage{amsmath,amsthm,latexsym,paralist}

\theoremstyle{definition}
\newtheorem{problem}{Problem}
\newtheorem*{solution}{Solution}
\newtheorem*{resources}{Resources}

\newcommand{\name}[1]{\noindent\textbf{Name: #1}}
\newcommand{\honor}{\noindent On my honor, as an Aggie, I have neither
  given nor received any unauthorized aid on any portion of the
  academic work included in this assignment. Furthermore, I have
  disclosed all resources (people, books, web sites, etc.) that have
  been used to prepare this homework. \\[1.5ex]
 \textbf{Signature:} \underline{\hspace*{5cm}} }

 
\newcommand{\checklist}{\noindent\textbf{Checklist:}
\begin{compactitem}[$\Box$] 
\item Did you add your name? 
\item Did you disclose all resources that you have used? \\
(This includes all people, books, websites, etc.\ that you have consulted)
\item Did you sign that you followed the Aggie honor code? 
\item Did you solve all problems? 
\item Did you submit (a) your latex source file and (b) the resulting pdf file
  of your homework on csnet?
\item Did you submit (c) a signed hardcopy of the pdf file in class? 
\end{compactitem}
}

\newcommand{\problemset}[1]{\begin{center}\textbf{Problem Set #1}\end{center}}
\newcommand{\duedate}[2]{\begin{quote}\textbf{Due dates:} Electronic
    submission of hw6.tex and hw6.pdf files of this homework is due on
    \textbf{#1} on csnet.cs.tamu.edu.  Please do not archive or
    compress the files.  A signed paper copy of the pdf file is due on
    \textbf{#2} at the beginning of class.\end{quote}}
    


\begin{document}
\vspace*{-15mm}
\begin{center}
{\large
CSCE 222-501 Discrete Structures for Computing\\[.5ex]
Fall 2014 -- Hyunyoung Lee\\}
\end{center}
\problemset{6}
\duedate{10/27/2014 before 23:59}{10/28/2014} 
\name{Eric E. Gonzalez}
\begin{resources} Discrete Mathematics and its Applications 7th Ed.(Rosen)
\end{resources}
\honor

\bigskip

\begin{problem} 
Section 5.3, Exercise 4, page 357
\end{problem}
\begin{solution} 
$\break$
$f(0)=f(1)=1$
\\
\\(a) $f(n+1)=f(n)-f(n-1)$
\\$f(2)=f(1)-f(0)$
\\$=1-1$
\\$=0$
\\$f(3)=f(2)-f(1)$
\\$=0-1$
\\$=-1$
\\$f(4)=f(3)-f(2)$
\\$=-1-0$
\\$=-1$
\\$f(5)=f(4)-f(3)$
\\$=-1-(-1)$
\\$=0$
\\
\\(b) $f(n+1)=f(n)f(n-1)$
\\$f(2)=f(1)f(0)$
\\$=1*1$
\\$=1$
\\$f(3)=f(2)f(1)$
\\$=1*1$
\\$=1$
\\$f(4)=f(3)f(2)$
\\$=1*1$
\\$=1$
\\$f(5)=f(4)f(3)$
\\$=1*1$
\\$=1$
\\
\\(c) $f(n+1)=f(n)^2 + f(n-1)^3$
\\$f(2)=f(1)^2 + f(0)^3$
\\$=1^2 + 1^3$
\\$=1 + 1$
\\$=2$
\\$f(3)=f(2)^2 + f(1)^3$
\\$=2^2 + 1^3$
\\$=4 + 1$
\\$=5$
\\$f(4)=f(3)^2 + f(2)^3$
\\$=5^2 + 2^3$
\\$=25 + 8$
\\$=33$
\\$f(5)=f(4)^2 + f(3)^3$
\\$=33^2 + 5^3$
\\$=1089 + 125$
\\$=1214$
\\
\\(d)
\\$f(n+1)=f(n)/f(n-1)$
\\$f(2)=f(1)/f(0)$
\\$= 1/1 = 1$
\\$f(3)=f(2)/f(1)$
\\$= 1/1 = 1$
\\$f(4)=f(3)/f(2)$
\\$= 1/1 = 1$
\\$f(5)=f(4)/f(3)$
\\$= 1/1 = 1$
\end{solution}

\begin{problem} 
Section 5.3, Exercise 6, page 357
\end{problem}
\begin{solution} 
$\break$
(a) The function is well-defined
\\ $f(n)=-f(n-1)$ for $ n \ge 1$.
\\Basis step: $f(0) = 1 = (-1)^0$
\\Induction step: $f(n)=-f(n-1)$
\\$=-(-1)^{n-1}$
\\$=(-1)^n$
\\Therefore, claim proven by induction.
\\(b) The function is well-defined
\\$2^m$ if $n=3k$,
\\$0$ if $n=3k+1$, 
\\and $2^{m+1} $ if $n=3k+2$
\\for some integer $k$.
\\Basis step:
\\$f(0)=1=2^0$
\\$f(1)=0$
\\$f(2)=2=2^{0+1}$
\\Induction step: $f(n)=2f(n-3)$ for $n\ge3$
\\Therefore, $f(n)$ = $2^k$, or, = $0$, or, = $2^{k+1}$
\\The claim is proven by strong induction.
\\c) Function is invalid because no base case can be reached.
\\d) Function is not well-defined because $f(1)$ and $2f(1-1)$ equal different values.
\\e) The function is defined by $f(n)=2^{\lfloor n/2\rfloor+1}$
\\Basis Step:
\\$f(0)=2=2^1=2^{\lfloor0/2\rfloor+1}$
\\$f(1)=f(0)=2=2^1=2^{\lfloor1/2\rfloor+1}$
\\Induction Step:
\\$f(n)=2f(n-2)$
\\$=2*2^{\lfloor(n-2)/2\rfloor+1}$
\\$=2*2^{\lfloor n/2 \rfloor}= 2^{\lfloor n/2 \rfloor + 1}$
\\Thus, the claim is proven by induction.
\end{solution}

\begin{problem} 
Section 5.3, Exercise 8, page 358
\end{problem}
\begin{solution} 
$\break$
a)  $a_1=2$ and $a_n=4 + a_{n-1}$ for $n \ge 2$.
\\b)  $a_1=0$, $a_2=2$, and $a_n= a_{n-2}$ for $n \ge 3$.
\\c)  $a_1=2$ and $a_n=a_{n-1} + 2n$ for $n \ge 2$.
\\d)  $a_1=1$ and $a_n= a_{n-1}+2n-1$ for $n \ge 2$.
\end{solution}

\begin{problem} 
Section 5.3, Exercise 12, page 358
\end{problem}
\begin{solution} 
$\break$
Basis Step: $P(1)=f_1^2=f_1f_1=1*1=1$
\\Induction Step: 
\\$P(k+1)=f_1^2+f_2^2+...+f_k^2+f_{k+1}^2=f_kf_k+1+f_{k+1}^2$
\\$=f_kf_{k+1}+f_{k+1}f_{k+1}$
\\$=f_{k+1}[f_k+f_{k+1}]$
\\$=f_{k+1}f_{k+2}$
\\Therefore, true by induction.
\end{solution}

\begin{problem} 
Section 5.3, Exercise 14, page 358
\end{problem}
\begin{solution} 
$\break$
Basis Step: $P(1)=f_{1+1}f_{1-1}-f_1^2=f_2f_0-f_1f_1=1*0-1*1=(-1)^1$
\\Induction Step:
\\$P(k+1)=f_{(k+1)+1}f_{(k+1)-1}-f_{k+1}^2$
\\$=f_{k+2}f_k-f_{k+1}^2$
\\$=f_{k+2}f_k-f_{k+1}f_{k+1}$
\\$=(f_{k+1}+f_k)f_k-f_{k+1}f{k+1}$
\\$=f_{k+1}f_k+f_k^2-f_{k+1}[f_k+f_{k-1}]$
\\$=f_{k+1}f_k+f_k^2-f_{k+1}f_k-f_{k+1}f_{k-1}$
\\$f_k^2-f_{k+1}f_{k-1}$
\\$=-(-1)^k$
\\$=-1(-1)^k$
\\$=(-1)^{k+1}$
\\Therefore, proven by induction.

\end{solution}

\begin{problem} 
Section 5.3, Exercise 16, page 358
\end{problem}
\begin{solution} 
$\break$
Basis Step: $f_0-f_1+f_2=0-1+1=1-1=f_1-1$
\\Induction Step:  
\\$\Sigma^{2n+2}_{k=0}(-1)^kf_k$
\\=$f_{2n-1} -1 - f_{2n+1}+f_{2n+2}$
\\=$f_{2n-1}+f_{2n}-1$
\\=$f_{2n+1}-1$
\\=$f_{2(n+1)-1}-1$
\\Therefore, claim proven by induction.
\end{solution}

\begin{problem} 
Section 5.3, Exercise 20, page 358 
\end{problem}
\begin{solution} 
$\break$
max:
\\$max(a_i,a_{i+1})=a_i if a_i \ge a_{i+1} for i=0,1,2,...,n-1$
\\$max(a_1,a_2,...,a_n)=max(max(a_1,a_2,...,a_{n-1}),a_n)$
\\
\\min:
\\$min(a_i,a_{i+1})=a_i if a_i \le a_{i+1} for i=0,1,2,...,n-1$
\\$min(a_1,a_2,...,a_n)=min(min(a_1,a_2,...,a_{n-1}),a_n)$
\end{solution}

\begin{problem} 
Section 5.3, Exercise 26, page 358
\end{problem}
\begin{solution} 
$\break$
(a)
\\1st Application: $(2,3),(3,2)$
\\2nd Application: $(4,6),(5,5),(6,4)$
\\3rd Application: $(6,9),(7,8),(8,7),(9,6)$
\\4th Application: $(8,12),(9,11),(10,10),(11,9),(12,8)$
\\5th Application: $(10,15),(11,14),(12,13),(13,12),(14,11),(15,11)$
\\
\\(b)
\\Basis Step:
\\$5|a+b$ with $(0,0)$
\\$5|0+0$, so basis holds
\\Inductive Hypothesis: Assume 5|a+b
\\Recursive Step: 
\\True by strong induction. Cause I ran out of time.
\\
\\(c)
\\Basis Step:
\\$5|a+b$ when a=0 and b=0
\\$5|0$, so basis holds
\\Inductive Hypothesis: Assume $5|a+b$ for $(a,b) \in S$
\\Recursive Step:
\\Show that (a+2, b+3) and (a+3, b+2) are divisible by 5
\\5|a+b is true by I.H.
\\5|a+2+b+3 = 5|a+b+5
\\5|5
\\Thus, we can prove the claim by structural induction.
\end{solution}

\begin{problem} 
Section 5.3, Exercise 36, page 359
\end{problem}
\begin{solution} 
$\break$
Let $w_1$ = ab
\\Let $w_2$ = cd
\\$w_1w_2$ = abcd
\\$(w_1w_2)^R$ = dcba
\\
\\Recursive Step:
\\$w_1^R$ = ba
\\$w_2^R$ = dc
\\$w_2^Rw_1^R$ = dcba
\\= $(w_1w_2)^R$
\end{solution}

\begin{problem} 
Section 5.3, Exercise 44, page 359 
\end{problem}
\begin{solution} 
$\break$
Let T be a binary tree with a single root node.
\\$l(T)=1, i(T)=0$
\\$l(T) = l(T_1) + l(T_2)$
\\$= (i(T_1) + 1) + (i(T_2) + 1)$
\\As such, for any binary tree T, $l(T) = i(T) + 1$
\\The claim is proven by structural induction.
\end{solution}

\bigskip

\goodbreak
\checklist
\end{document}
