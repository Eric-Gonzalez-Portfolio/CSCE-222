\documentclass{article}
\usepackage{amsmath,amsthm,latexsym,paralist,soul}

\theoremstyle{definition}
\newtheorem{problem}{Problem}
\newtheorem*{solution}{Solution}
\newtheorem*{resources}{Resources}

\newcommand{\name}[1]{\noindent\textbf{Name: #1}}
\newcommand{\honor}{\noindent On my honor, as an Aggie, I have neither
  given nor received any unauthorized aid on any portion of the
  academic work included in this assignment. Furthermore, I have
  disclosed all resources (people, books, web sites, etc.) that have
  been used to prepare this homework. \\[2ex]
 \textbf{Signature:} \underline{\hspace*{7cm}} }

 
\newcommand{\checklist}{\noindent\textbf{Checklist:}
\begin{compactitem}[$\Box$] 
\item Did you add your name? 
\item Did you disclose all resources that you have used? \\
(This includes all people, books, websites, etc. that you have consulted)
\item Did you sign that you followed the Aggie honor code? 
\item Did you solve all problems? 
\item Did you submit (a) your latex source file and (b) the resulting pdf file
  of your homework on csnet?
\item Did you submit (c) a signed hardcopy of the pdf file in class? 
\end{compactitem}
}

\newcommand{\problemset}[1]{\begin{center}\textbf{Problem Set #1}\end{center}}
\newcommand{\duedate}[2]{\begin{quote}\textbf{Due dates:} Electronic
    submission of hw7.tex and hw7.pdf files of this homework is due on
    \textbf{#1} on csnet.cs.tamu.edu.  Please do not archive or
    compress the files.  A signed paper copy of the pdf file is due on
    \textbf{#2} at the beginning of class.\end{quote} }


\begin{document}
\vspace*{-15mm}
\begin{center}
{\large
\textbf{***Three problems are discarded and three new problems are added in boldface.***}\\[1ex]
CSCE 222-501 Discrete Structures for Computing\\[.5ex]
Fall 2014 -- Hyunyoung Lee\\}
\end{center}
\problemset{7}
\duedate{11/3/2014 before 23:59}{11/4/2014}

\name{Eric E. Gonzalez}
\begin{resources} Discrete Mathematics and its Applications 7th Ed.(Rosen)
\end{resources}
\honor

\bigskip
\begin{problem}
\textbf{Section 6.1, Exercise 14, page 396}
\end{problem}
\begin{solution} 
$\break$
The number of strings with length $n = 2^n$, providing $2^n$ numbers of choices. 
\\The number of bit strings with 1 at the beginning and end = $n - 2$ choices.
\\The number of strings with 1s at the beginning and end is $2^{n-2}$.
\end{solution}

\begin{problem}
\textbf{Section 6.1 Exercise 16, page 396}
\end{problem}
\begin{solution} 
$\break$
Let $S$ = the number of possible strings with x
\\Let $T$ = the number of possible strings without x
\\$S = 26^4 = 456976$
\\$T = 25^4 = 390625$
\\$S - T = 456976 - 390625$
\\$= 66351$
\end{solution}

\begin{problem}
Section 6.1, Exercise 32, page 397
\end{problem}
\begin{solution}
$\break$ 
(a) $26^8$
\\(b) $26*25*24*23*22*21*20*19$
\\(c) $25^7$
\\(d) $25*24*23*22*21*20*19$
\\(e) $25^6$
\\(f) $25^6$
\\(g) $25^4$
\\(h) $2(26^6) - 26^4$
\end{solution}

\begin{problem} 
Section 6.1, Exercise 70, page 398
\end{problem}
\begin{solution} 
$\break$
$n = 2^n$ by the Product Rule
\\So, $2^n = 2^{2^n}$
\\Thus, there are $2^{2^n}$ different truth tables for $n$ variables.
\end{solution}

\begin{problem} 
Section 6.2, Exercise 8, page 405
\end{problem}
\begin{solution}
$\break$
$f: S \rightarrow T$ and $|S| \ge |T|$
\\$|T| = n$ and $|S| \ge n+1$
\\As such, n+1 elements are set to map to n elements.
\\So, by pigeon-hole principle, at least one pair of elements maps to the same spot, making $f(s_1) = f(s_2)$.
\\Therefore, the function $f$ is not one-to-one.
\end{solution}

\begin{problem}
Section 6.2, Exercise 16, page 405
\end{problem}
\begin{solution} 
$\break$
If we arrange the set into pairs that add to 16, then we get (1,15),(3,13),(5,11),(7,9).
\\Since there are 4 pairs of numbers that add to 16, we need to choose at least 5 to ensure that two numbers will be a matching pair and add up to 16.
\end{solution}

\begin{problem} 
Section 6.2, Exercise 46, page 407
\end{problem}
\begin{solution} 
$\break$
Assume each box contains at most $n_i - 1$ objects.
\\If $n$ is the maximum number of objects that can be placed into a $t$ number of boxes, then:
\\$n = n_1 - 1 ...+ n_t - 1$
\\$= n_1 ... + ...n_t - t$
\\$< n_t - t +1$
\\It is not possible to place all $n_t - t +1 $ into the $t$ boxes. As such, for some $i$, the $i$th box must contain at least $n_i$ objects.
\end{solution}

\begin{problem}
Section 6.3, Exercise 12, page 413
\end{problem}
\begin{solution} 
$\break$
(a) $12 \choose 3$ = 220
\\(b) $12 \choose 0$ + $12 \choose 1$ + $12 \choose 2$ + $12 \choose 3$ = 299
\\(c) $12 \choose 3$ + $12 \choose 4$ + $12 \choose 5$ + $12 \choose 6$ + $12 \choose 7$ + $12 \choose 8$ + $12 \choose 9$ + $12 \choose 10$ + $12 \choose 11$ + $12 \choose 12$ = 4017
\\(d) $12 \choose 6$ = 924
\end{solution}

\begin{problem}
\textbf{Section 6.3 Exercise 18 on page 413}
\end{problem}
\begin{solution} 
$\break$
(a) $2^8 = 256$
\\(b) $8 \choose 3$ = $56$
\\(c) $8 \choose 5$ + $8 \choose 4$ + $8 \choose 3$ + $8 \choose 2$ + $8 \choose 1$ + $8 \choose 0$ = 219
\\(d) $8 \choose 4$ = $70$
\end{solution}

\begin{problem} 
Section 6.3, Exercise 22, page 414
\end{problem}
\begin{solution} 
$\break$
(a) $7! = 5040$
\\(b) $6! = 720$
\\(c) $5! = 120$
\\(d) $5! = 120$
\\(e) $4! = 24$
\\(f) 0, as there are no permutations available. B can't be followed by both A and F simultaneously.
\end{solution}

\ \\
\noindent
%\begin{problem} 
\st{Section 6.4, Exercise 14, page 421}
%\end{problem}
%\begin{solution} 
%\end{solution}

\ \\
\noindent
%\begin{problem} 
\st{Section 6.4, Exercise 16, page 421}
%\end{problem}
%\begin{solution} 
%\end{solution}

\ \\
\noindent
%\begin{problem} 
\st{Section 6.4, Exercise 38, page 422 (read the hint carefully!)}
%\end{problem}
%\begin{solution} 
%\end{solution}
\ \\

\goodbreak
\checklist
\end{document}
