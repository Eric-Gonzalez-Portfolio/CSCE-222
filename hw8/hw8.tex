\documentclass{article}
\usepackage{amsmath,amsthm,latexsym,paralist}

\theoremstyle{definition}
\newtheorem{problem}{Problem}
\newtheorem*{solution}{Solution}
\newtheorem*{resources}{Resources}

\newcommand{\name}[1]{\noindent\textbf{Name: #1}}
\newcommand{\honor}{\noindent On my honor, as an Aggie, I have neither
  given nor received any unauthorized aid on any portion of the
  academic work included in this assignment. Furthermore, I have
  disclosed all resources (people, books, web sites, etc.) that have
  been used to prepare this homework. \\[2ex]
 \textbf{Signature:} \underline{\hspace*{7cm}} }

 
\newcommand{\checklist}{\noindent\textbf{Checklist:}
\begin{compactitem}[$\Box$] 
\item Did you add your name? 
\item Did you disclose all resources that you have used? \\
(This includes all people, books, websites, etc. that you have consulted)
\item Did you sign that you followed the Aggie honor code? 
\item Did you solve all problems? 
\item Did you submit (a) your latex source file and (b) the resulting pdf file
  of your homework on csnet?
\item Did you submit (c) a signed hardcopy of the pdf file in class? 
\end{compactitem}
}

\newcommand{\problemset}[1]{\begin{center}\textbf{Problem Set #1}\end{center}}
\newcommand{\duedate}[2]{\begin{quote}\textbf{Due dates:} Electronic
    submission of hw8.tex and hw8.pdf files of this homework is due on
    \textbf{#1} on csnet.cs.tamu.edu.  Please do not archive or
    compress the files.  A signed paper copy of the pdf file is due on
    \textbf{#2} at the beginning of class.\end{quote} }


\begin{document}
\vspace*{-15mm}
\begin{center}
{\large
CSCE 222-501 Discrete Structures for Computing\\[.5ex]
Fall 2014 -- Hyunyoung Lee\\}
\end{center}
\problemset{8}
\duedate{11/17/2014 before 23:59}{11/18/2014}

\name{ Eric E. Gonzalez}
\begin{resources}  Discrete Mathematics and its Applications 7th Ed.(Rosen)
\end{resources}
\honor

\bigskip

\noindent
In this problem set, you will earn total $100+20$ (extra credit) points.

\begin{problem} (10 points)
Section 8.1, Exercise 2, page 510
[Hint: Let $P_n$ denote the number of permutations of a set with $n$ elements.
The initial condition is $P_0 = 1$.] 
\end{problem}
\begin{solution} 
$\break$
a) Let pn be the number of permutations of a set with n elements. We
can make a recurrence relation choosing a location for the first element, then permuting the remaining n - 1 elements. Thus, p(n) = np(n - 1).
\\b) p(1) = 1
\\p(2) = 2
\\p(3) = 6
\\Iterating, we see p(n) = n(n - 1)(n - 2)$\ldots$(2)p(1)
\\ so p(n) = n!
\end{solution}

\begin{problem} (10 points)
Section 8.1, Exercise 12, page 511
[Hint: Let $S_n$ denote the number of ways to climb $n$ stairs.
One of the initial conditions is $S_0 = 1$.] 
\end{problem}
\begin{solution} 
$\break$
a) $a_n = a_{n-1} + a_{n-2} + a_{n-3}$ for $n \ge 3$
\\b) $a_0 = 1; a_1 = 1; a_2 = 2$
\\c) The number of ways one can climb the 8 flights of stairs: $a_8$
\\Sequence: $a_3 = 4; a_4 = 7; a_5 = 13; a_6 = 24; a_7 = 44; a_8 = 81$
\end{solution}

\begin{problem} (10 points)
Section 8.1, Exercise 20, page 511
\end{problem}
\begin{solution} 
$\break$
a) Recurrence Relation: $a_{5n}=a_{5(n-1)}+a_{5(n-2)} for n \ge 2$
\\Initial Conditions: $a_0=1; a_5=1$
\\b) Number of different ways driver can pay toll: $a_45$
\\Sequence: $a_{10}=2$; $a_{15}=3$; $a_{20}=5$; $a_{25}=8$; $a_{30}=13$; $a_{35}=21$; $a_{40}=34$; $a_{45}=55$
\end{solution}

\begin{problem} (10 points)
Section 8.1, Exercise 28, page 512. This problem has two parts as below. 
\end{problem}
\begin{solution} 
\ \\
a) (4 points) Show that the Fibonacci numbers satisfy $\ldots$
\ \\For $n \ge 5$,
\\$fn = f_{n-1} + f_{n-2}$
\\$= (f_{n-2} + f_{n-3}) + (f_{n-3} + f_{n-4})$
\\$= f_{n-3} + f_{n-4} + 2f_{n-3} + f_{n-4}$
\\$= 3f_{n-3} + 2f_{n-4}$
\\$= 3f_{n-4} + 3f_{n-5} + 2f_{n-4}$
\\$= 5f_{n-4} + 3f_{n-5}$

b) (6 points) Use this recurrence relation to show that $\ldots$ (prove by induction
on $n$)
\ \\Since $f_5 = f_3 + f_4 = 5$, P(1) is true. Assume that P(n) is true. $f_{5n} = 5k$ for some integer k.
\\Then, $f_{5(n+1)} = f_{5n+5}$
\\$= 5f_{5(n+1)−4} + 3f_{5(n+1)−5}$
\\$= 5f_{5n+1} + 3f_{5n}$
\\$= 5f_{5n+1} + 15k$
\\$= 5(f_{5n+1} + 3k)$
which is divisible by 5. Therefore, P (N+1) is true.
\end{solution}

\begin{problem} (15 points)
Section 8.1, Exercise 32 a), b), c) and d), page 512
\end{problem}
\begin{solution} 
$\break$
a) The number of moves needed to move n disks from peg 1 to peg 3 is $a_n$. With just one disk, one must move it to peg 2, then peg 3. Thus, $a_1 = 2$. With n+1 disks, one must move n disks from peg 1 to peg 3, then move the largest disk to peg 2, and then move the n disks from peg 3 back to peg 1 so that one may move the largest disk to peg 3. Then, one may finally move the n disks back to peg 3, giving a recurrence relation of $a_n = 3a_{n-1} + 2$
\\b) Assume $a_n=3^n-1.$
\\Base step: $a_1=2=3^1-1$
\\Inductive step: $a_{n+1}=3a_n+2=3(3^n-1)+2=3^{n+1}-1$
\\Thus, the claim is proven by induction.
\\c) By the product rule, there are $3^n$ different methods of disc placement.
\\d) According to part (b), there are $3^n$ different arrangements.
\end{solution}

\begin{problem} (10 points)
Section 8.2, Exercise 4 c), d), e) and f), page 524
\end{problem}
\begin{solution} 
$\break$
c) Characteristic equation: $r^2-6r+8=0$
\\Solution: $\alpha_n = 3(2^n)+1(4^n)$
\\d) Characteristic equation: $r^2-2r+1=0$
\\Solution: $\alpha_n = 4-3n$
\\e) Characteristic equation: $r^2-0r-1=0$
\\Solution: $\alpha_n = 2(1^n)+3(-1)^n$
\\f) Characteristic equation: $r^2+4r-5=0$
\\Solution: $\alpha_n = 3(-3^n)-2n(-3)^n$
\end{solution}

\begin{problem} (10 points)
Section 8.2, Exercise 8, page 524--525
\end{problem}
\begin{solution} 
$\break$
a) $L_n$ is the number of lobsters caught in year $n$ for the model of $L_n = (1/2)L_{n-1} + (1/2)L_{n-2} $ 
\\b) Characteristic equation: $r^2 - (1/2)r - (1/2) = 0= (1/2)(2r+1)(r-1)$
\\Roots: $r_1= -(1/2)$ and $r_2 = 1$
\\Solution: $r_1= -(1/2)$ and $r_2 = 1$, so $a_n = k_1 (-1/2)^n + k_2$.
\\Thus, $(-1/2)k_1 + k_2 = 100000$, and $(1/4)k_1 + k_2 = 300000.$ 
\\Solving for system of equations gives $k_1= 800000/3$, $k_2 = 700000/3$.
\\Therefore, $a_n = (800000/3)(-1/2)^n + (700000/3).$
\end{solution}

\begin{problem} (10 points)
Section 8.4, Exercise 6 a), b), c) and d), page 549
\end{problem}
\begin{solution} 
$\break$
a) Generating function of $(-1)_n$ is $1/(x-1)$.
\\b) Generating function of $(2^n)_{n \ge 1}, a_0=1$ is $2x/(1-2x).$
\\c) Generating function of $(n-1)_n$ is $-1+x^2/(1-x)^2.$
\\d) Generating function of $(1/(n+1)!)_n$ is $(e^x-1)/x.$
\end{solution}

\begin{problem} (15 points)
Section 8.4, Exercise 8 a), b) and c), page 549
\end{problem}
\begin{solution} 
$\break$
a) $x^6+3x^4+3x^2+1$
\\$(a_0,a_1,a_2,a_3,a_4,a_5,a_6,a_7,...)=(1,0,3,0,3,0,1,0,...)$
\\$a_n=0$ for all $n \ge 7$
\\b) $(3x-1)^3=27x^3-27x^2+9x-1$
\\$(a_0,a_1,a_2,a_3,a_4,...)=(-1,9,-27,27,0,...)$
\\$a_n=0$ for all $n \ge 4$
\\c) Sequence for generating function of $1/(1-2x^2)$:
\\$a_{2n}=2^n$
\\$a_{2n+1}=0$ for all $n \ge 0$
\end{solution}

\begin{problem} (Extra credits: 5 points)
Section 8.4, Exercise 6 e) and f), page 549
\end{problem}
\begin{solution} 
$\break$
e) $a_n=(^n_2)$
\\Generating function: $x^2/(1-x)^3$
\\f) $a_n=(^{10}_{n+1})$
\\Generating function: $(-1+(1+x)^{10})/x$
\end{solution}

\begin{problem} (Extra credits: 15 points)
Section 8.4, Exercise 8  d), e) and f), page 549
\end{problem}
\begin{solution} 
$\break$
d) Sequence for generating function of $x^2/(1-x)^3$: $a_n=(^n_2)$
\\e) Sequence for generating function of $4x+9x^2+27x^3+81x^4+...$: $a_0=0, a_1=4$, and $a_n=3^n$ for all $n \ge 2$
\\f) Sequence for generating function of $(1+x^3)/(1+x)^3$: $a_0=1$ and $a_n=(-1)^n*3n$ for all $n \ge 1$
\end{solution}

\goodbreak
\checklist
\end{document}
