\documentclass{article}
\usepackage{amsmath,amsthm,latexsym,paralist}
\input xypic

\theoremstyle{definition}
\newtheorem{problem}{Problem}
\newtheorem*{solution}{Solution}
\newtheorem*{resources}{Resources}

\newcommand{\name}[1]{\noindent\textbf{Name: #1}}
\newcommand{\honor}{\noindent On my honor, as an Aggie, I have neither
  given nor received any unauthorized aid on any portion of the
  academic work included in this assignment. Furthermore, I have
  disclosed all resources (people, books, web sites, etc.) that have
  been used to prepare this homework. \\[2ex]
 \textbf{Signature:} \underline{\hspace*{7cm}} }

 
\newcommand{\checklist}{\noindent\textbf{Checklist:}
\begin{compactitem}[$\Box$] 
\item Did you add your name? 
\item Did you disclose all resources that you have used? \\
(This includes all people, books, websites, etc. that you have consulted)
\item Did you sign that you followed the Aggie honor code? 
\item Did you solve all problems? 
\item Did you submit (a) your latex source file and (b) the resulting pdf file
  of your homework on csnet?
\item Did you submit (c) a signed hardcopy of the pdf file in class? 
\end{compactitem}
}

\newcommand{\problemset}[1]{\begin{center}\textbf{Problem Set #1}\end{center}}
\newcommand{\duedate}[2]{\begin{quote}\textbf{Due dates:} Electronic
    submission of hw9.tex and hw9.pdf files of this homework is due on
    \textbf{#1} on csnet.cs.tamu.edu. A signed paper copy of the pdf file is due on
    \textbf{#2} at the beginning of class.\end{quote} }


\begin{document}
\vspace*{-15mm}
\begin{center}
{\large
CSCE 222-501 Discrete Structures for Computing\\[.5ex]
Fall 2014 -- Hyunyoung Lee\\}
\end{center}
\problemset{9}
\duedate{Monday 11/24/2014 before 23:59}{Tuesday 11/25/2014}

\name{Eric E. Gonzalez}
\begin{resources} Discrete Mathematics and its Applications 7th Ed.(Rosen)
\end{resources}
\honor

\bigskip

\noindent
In this problem set, you will earn total $100+15$ (extra credit) points. Each question
is worth 10 points unless otherwise noted.

\begin{problem} 
Chapter 9.1, Exercise 4, page 581
\end{problem}
\begin{solution} 
$\break$
a) Antisymmetric, transitive
\\b) Reflexive, symmetric, transitive
\\c) Reflexive, symmetric, transitive
\\d) Reflexive, symmetric
\end{solution}

\begin{problem} (15 points)
Chapter 9.1, Exercise 6, page 581
\end{problem}
\begin{solution} 
$\break$
a) $(1,1)\notin R.$ \qquad Not reflexive
\\If $x+y=0$, then $y+x=0$. \qquad Symmetric
\\$(1,-1)\in R$ and $(-1,1)\in R$. $1\not=-1.$ \qquad Not anti-symmetric
\\$(2,-2)\in R$ and $(-2,2)\in R$. $(2,2)\notin R.$ \qquad Not transitive
\\ \\b) $x=x$ for all real numbers. \qquad Reflexive
\\$x= \pm y$ and $y=\pm x$. \qquad Symmetric
\\$(1,-1)\in R$ and $(-1,1)\in R$. $1\not=-1.$ \qquad Not anti-symmetric
\\If $x= \pm y$ and  $y=\pm  z$, $x=\pm z.$  \qquad Transitive
\\ \\c) $x-x=0$ is rational.  \qquad Reflexive
\\$x-y$ and $y-x$ are both rational. \qquad Symmetric
\\$(2,3)\in R$ and $(3,2)\in R$. $2\not=3.$ \qquad Not anti-symmetric
\\$x-y$ is rational and $y-z$ is rational, therefore $x-z$ is rational. \qquad Transitive
\\ \\d) $(1,1)\notin R.$ \qquad Not reflexive
\\$(1,2) \in R$, $(2,1) \notin R$. \qquad Not symmetric
\\$x=0=y$.\qquad Anti-symmetric.
\\$(8,4) \in R$ and $(4,2)\in R$, however $(2,4)\notin R$.\qquad Not transitive
\\ \\e) $x^2 \ge 0$ for all real numbers. \qquad Reflexive
\\$xy = yx$. \qquad Symmetric
\\$(4,2)\in R$ and $(2,4)\in R$. $2\not=4.$ \qquad Not anti-symmetric
\\$(-2,0) \in R$ and  $(0,4)\in R$, however $(-2,4)\notin R$. \qquad Not transitive
\\ \\f) $(1,1)\notin R.$ \qquad Not reflexive
\\$xy=yx$. \qquad Symmetric
\\$(1,0)\in R$ and $(0,1)\in R$. $1\not=0.$ \qquad Not anti-symmetric
\\$(1,0) \in R$ and $(0,4)\in R$, however $(1,4) \notin R.$ \qquad Not transitive
\\ \\g) $(2,2)\notin R.$ \qquad Not reflexive
\\$(1,2) \in R$, $(2,1) \notin R$. \qquad Not symmetric
\\$x=y$.\qquad Anti-symmetric.
\\This function is not reflexive, symmetric, or transitive for any value of $x$ other than $1$.
\\ \\h) $(2,2)\notin R$ \qquad Not reflexive
\\$(y,x) \in R$ for any $(x,y)$ because either $x=1$ or $y=1$. \qquad Symmetric
\\$(1,2)\in R$ and $(2,1)\in R$. $1\not=2.$ \qquad Not anti-symmetric
\\$(2,1)\in R$ and $(1,2)\in R$, however $(2,2) \notin R$. \qquad Not transitive
\end{solution}

\begin{problem} 
Chapter 9.1, Exercise 42, page 583
\end{problem}
\begin{solution} 
$\break$
$R_{1}=\emptyset $
\\$R_{2}=\{(0,0)\}$\
\\$R_{3}=\{(0,1)\}$
\\$R_{4}=\{(1,0)\}$
\\$R_{5}=\{(1,1)\}$
\\$R_{6}=\{(0,0),(0,1)\}$
\\$R_{7}=\{(0,0),(1,0)\}$
\\$R_8=\{(0,0),(1,1)\}$
\\$R_9=\{(0,1),(1,0)\}$
\\$R_{10}=\{(0,1),(1,1)\}$
\\$R_{11}=\{(1,0),(1,1)\}$
\\$R_{12}=\{(0,0),(0,1),(1,0)\}$
\\$R_{13}=\{(0,0),(0,1),(1,1)\}$
\\$R_{14}=\{(0,0),(1,0),(1,1)\}$
\\$R_{15}=\{(0,1),(1,0),(1,1)\}$
\\$R_{16}=\{(0,0),(0,1),(1,0),(1,1)\}$
\end{solution}

\begin{problem} (20 points)
Chapter 9.1, Exercise 44, page 583
\end{problem}
\begin{solution} 
$\break$
a)Reflexive: $R_8,R_{13},R_{14},R_{16}$
\\b)Irreflexive: $R_1,R_3,R_9,R_4$
\\c)Symmetric: $R_1,R_2,R_5,R_8,R_9,R_{12},R_{15},R_{16}$
\\d)Anti-Symmetric: $R_1,R_2,R_3,R_4,R_5,R_6,R_7,R_8,R_{10},R_{11},R_{13},R_{14}$
\\e)Asymmetric: $R_3,R_4$
\\f)Transitive: $R_1,R_2,R_3,R_4,R_5,R_6,R_7,R_8,R_{10},R_{11},R_{13},R_{14},R_{16}$
\end{solution}

\begin{problem} 
Chapter 9.5, Exercise 2, page 615
\end{problem}
\begin{solution} 
$\break$
a) Equivalence relation
\\b) Equivalence relation
\\c) Not an equivalence relation: not transitive because though $a$ and $b$ may share a common parent and $b$ and $c$ may share a common parent, $a$ and $c$ do not necessarily share the same parent.
\\d) Not an equivalence relation: Not transitive. Though $a$ and $b$ may have met, and $b$ and $c$ may have met, $a$ and $c$ may have not met.
\\e) Not an equivalence relation: Not transitive. Though $a$ and $b$ may speak the same language, and $b$ and $c$ may speak the same language, $a$ and $c$ may not necessarily speak the same language.
\end{solution}

\begin{problem}
Chapter 9.5, Exercise 16, page 615
\end{problem}
\begin{solution} 
$\break$
$ab=ba$. \qquad Reflexive
\\$ad=bc$ is equivalent to $cb=da$.  \qquad Symmetric
\\$ad=bc$ and $cf=de$ 
\\$af=ad/df=ad*(f/d)=b/d*(cf)=b/d(de)=de.$ \qquad Transitive
\\As such, $R$ is an equivalence relation.
\end{solution}

\begin{problem} 
Chapter 9.5, Exercise 58, page 618
\end{problem}
\begin{solution} 
$\break$
a) The relation is reflexive because $B_1$ can be found by rotating $B_1$. Since $B_1$ can be obtained from $B_2$ by a composition of rotations, $(B_2,B_1)\in R$ for any $(B_1,B_2)\in R$. Thus, the relation is symmetric. From the composition of rotations, it is made clear that $R$ is transitive since $B_1,B_2$ and $B_3$ are in all in $R$.
\\b) $\{(B_1,B_1,B_1),(B_2,B_2,B_2),(B_3,B_3,B_3),(B_1,B_1,B_2),(B_1,B_1,B_3),
\\(B_2,B_2,B_1),(B_2,B_2,B_3),(B_3,B_3,B_1),(B_3,B_3,B_2)\}$
\end{solution}

\begin{problem} 
Chapter 9.6, Exercise 4, page 630
\end{problem}
\begin{solution} 
$\break$
a) $(S,R)$ is not a poset: $R$ is not antisymmetric because there may be two different people with the same height.
\\b) $(S,R)$ is not a poset: $R$ is not reflexive because it is not possible for a person to weigh more than themself.
\\c) $(S,R)$ is a poset. $R$ is reflexive since $(a,a)\in R$ for all $a\in S$. For some $(a,b) \in S, a\not= b$, and $(a,b)\in R$, $a$ is a descendant of $b$ and, therefore, $b$ cannot be a descendant of $a$. As such, $(b,a)\notin R$ and $R$ is antisymmetric. Finally, $R$ is transitive since $(a,b)\in R$ and $(b,c)\in R$ implies $(a,c)\in R$.
\\d) $(S,R)$ is not a poset: $R$ is not reflexive because a person has the same friends as themself.
\end{solution}

\begin{problem} 
Chapter 9.6, Exercise 22, page 631
\end{problem}
\begin{solution} 
$\break$
a) The Hasse diagram of set \{1, 2, 3, 4, 5, 6\} with divisibility condition is given by\\
$$
\diagram
4 & 6 \\
2  \uline  \urline  & 3 \uline   \uline  & 5 \\
&1 \urline \uline 
\enddiagram
$$
\\
b) The Hasse diagram of set \{3, 5, 7, 11, 13, 16, 17\} with divisibility condition is given by\\
$$
\diagram
0 & 0 & 0 & 0 & 0 & 0 & 0  \\
3 & 5 & 7 & 11 & 13 & 16 & 17 \\
\enddiagram
$$
\\
c) The Hasse diagram of set \{2, 3, 5, 10, 11, 15, 25\} with divisibility condition is given by\\
$$
\diagram
10 & 25 & 15 \\
2\uline  & 5 \ulline \uline \urline  & 3 \uline  & 11\\
\enddiagram
$$
\\
d) The Hasse diagram of set \{1, 3, 9, 27, 81, 243\} with divisibility condition is given by\\
$$
\diagram
243 \\
81\uline\\
27\uline\\
9\uline\\
3\uline\\
1\uline\\
\enddiagram
$$
\\
\end{solution}

\begin{problem}
Chapter 9.6, Exercise 48, page 632
\end{problem}
\begin{solution} 
$\break$
$S$ is reflexive since $A_1 \prec A_2$ for all integers. If $A_1 \prec A_2$ and $A_2\prec A_1$, then $A_1=A_2$. So, $S$ is also antisymmetric. Finally, $S$ is transitive because $A_1 \prec A_2$ and $A_2 \prec A_3$ implies that $A_1\prec A_3$. Therefore, $(S, \prec)$ is a poset. Since $A_1 \in(A_1,C_1)$ and $A_2 \in(A_2,C_2)$, $A_1$ must be a maximal and $A_2$ must be a minimal since there exist no $A_1 \in S$ and $A_2 \in S$ such that $A_1<A_2$. As such, $S$ is a lattice.
\end{solution}


\goodbreak
\checklist
\end{document}
